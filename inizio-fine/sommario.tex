% !TEX encoding = UTF-8
% !TEX TS-program = pdflatex
% !TEX root = ../tesi.tex

%**************************************************************
% Sommario
%**************************************************************
\cleardoublepage
\phantomsection
\pdfbookmark{Sommario}{Sommario}
\begingroup
\let\clearpage\relax
\let\cleardoublepage\relax
\let\cleardoublepage\relax

\chapter*{Sommario}

Il presente documento descrive il lavoro svolto durante il periodo di stage, della durata di circa 320 ore, dal laureando Michele Veronesi presso l'azienda IM4DIGITAL s.r.l.\\
Il focus del progetto di stage era lo studio di fattibilità di un sistema di navigazione per droni ricognitivi che decidesse la rotta in base allo spettro luminoso degli oggetti sorvolati.\\
In particolare, è stato necessario inizialmente identificare il dataset su cui fare lo studio, successivamente si è rivelata fondamentale la pulizia di questo.
Poi è stato richiesto di prendere in considerazione diversi modelli di machine learning e analizzare le prestazioni nel classificare gli oggetti tramite le loro firme spettrali,
in modo da selezionare il più adatto.
Infine il modello allenato è stato caricato ed esposto su infrastruttura cloud, con cui si è interfacciata la business logic che effettua il calcolo del percorso (anch'essa realizzata nel corso del tirocinio).\\
La suddivisione in capitoli è la seguente:
\begin{description}
    \item[{\hyperref[cap:introduzione]{Il primo capitolo}}] presenta la realtà aziendale e gli obiettivi dello stage.
    
    \item[{\hyperref[cap:machine-learning]{Il secondo capitolo}}] descrive nel dettaglio la parte di machine learning e analisi dei dati.
    
    \item[{\hyperref[cap:business-logic]{Il terzo capitolo}}] racconta la realizzazione della business logic per il calcolo della traiettoria del drone.
    
    \item[{\hyperref[cap:conclusione]{Il quarto capitolo}}], infine, porta una valutazione retrospettiva delle attività svolte e alcune considerazioni finali.
\end{description}
%\vfill
%
%\selectlanguage{english}
%\pdfbookmark{Abstract}{Abstract}
%\chapter*{Abstract}
%
%\selectlanguage{italian}

\endgroup			

\vfill

