% !TEX encoding = UTF-8
% !TEX TS-program = pdflatex
% !TEX root = ../tesi.tex

%**************************************************************
% Sommario
%**************************************************************
\cleardoublepage
\phantomsection
\pdfbookmark{Sommario}{Sommario}
\begingroup
\let\clearpage\relax
\let\cleardoublepage\relax
\let\cleardoublepage\relax

\chapter*{Sommario}

Il presente documento descrive il lavoro svolto durante il periodo di stage, della durata di circa 320 ore, dal laureando Michele Veronesi presso l'azienda IM4DIGITAL s.r.l.\\
Il focus del progetto di stage era la creazione di un sistema di navigazione per droni senza pilota, i quali devono sorvolare dei campi di frutta e verdura raccogliendo informazioni
circa il grado di maturazione degli ortaggi.\\
In primo luogo era richiesto lo sviluppo della business logic, la quale si interfaccia con il drone per raccogliere dati sul tipo di terreno sorvolato. Inoltre deve interfacciarsi con
alcune API offerte da Google Maps in seguito descritte nel dettaglio. Dopo aver calcolato il prossimo percorso da far seguire al drone, deve quindi caricarlo su quest'ultimo.\\
In secondo luogo era richiesta l'implementazione di un modello di machine learning che elaborasse i dati raccolti dal drone, i quali sono le features che descrivono ogni esempio. In uscita
da questo è necessario dare il tipo di terreno sottostante (classificazione con \textit{k} classi) e quindi la frequenza di sorvolo in quella zona.\\
Terzo ed ultimo obiettivo era la definizione di un algoritmo per il controllo periodico delle zone scartate dalla classificazione, per rilevare eventuali cambiamenti.

%\vfill
%
%\selectlanguage{english}
%\pdfbookmark{Abstract}{Abstract}
%\chapter*{Abstract}
%
%\selectlanguage{italian}

\endgroup			

\vfill

