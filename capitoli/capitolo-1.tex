% !TEX encoding = UTF-8
% !TEX TS-program = pdflatex
% !TEX root = ../tesi.tex

%**************************************************************
\chapter{Introduzione}
\label{cap:introduzione}
%**************************************************************

\intro{Nel corso di questo capitolo si presenterà l'azienda in cui è stato svolto il progetto di stage,
portando inoltre l'attenzione sugli obiettivi, la pianificazione del lavoro e le modalità in cui si è svolto.}

%**************************************************************

\section{Contesto aziendale}

L'azienda ospitante è denominata IM4DIGITAL s.r.l., società neonata gemella di Trizeta s.r.l.\\
Infatti condividono locali, risorse e obiettivi. In particolare le due aziende operano a stretto contatto con le realtà del territorio,
al fine di favorire l'automatizzazione di vari processi in settori diversi da quello informatico.\\
Trizeta è una software house di Padova specializzata in Consulenza, Integrated Solutions e IT Services, dedicata all’ideazione e progettazione “in-house” di soluzioni tecnologiche. Dal 2012 sviluppa internamente soluzioni tecnologiche per automatizzare i processi industriali e accompagnare le imprese in ogni fase della loro Transizione 4.0, dall’ideazione alla messa in opera dei progetti informatici.
L’offerta di Trizeta favorisce il successo dei clienti attraverso la realizzazione di soluzioni specifiche e alla consolidata esperienza dei suoi collaboratori.
Dal sistema di controllo e gestione aziendale ADeMES a quello del magazzino ADeWMS, passando per la consulenza da remoto attraverso la realtà aumentata ADeGO, Trizeta offre tutti gli strumenti e i servizi all’avanguardia per portare le aziende nel cuore dell’industria 4.0.

\begin{figure}[!h]
	\centering 
	\includegraphics[width=0.5\textwidth]{trizeta.png}
	\caption{Logo Trizeta}
\end{figure}

Il team è composto da poche persone, circa una decina. La metodologia di lavoro prediletta è quella agile, in quanto si cerca di intrattenere un stretto rapporto con i clienti. Inoltre questi vengono affiancati costantemente nel processo di digitalizzazione aziendale.

%**************************************************************
\section{Progetto di stage}

Lo scopo principale dello stage è uno studio di fattibilità, seguito dalla realizzazione di una Proof of Concept, per un sistema di navigazione
per droni senza pilota, il cui obiettivo è quello di sorvolare un campo di frutta identificando le zone in cui le piante soffrono di malattie e lo stato
di maturazione delle colture.\\
In particolare parte del sistema deve essere composta da un modello di machine learning per classificare il tipo di oggetti sorvolati dal drone,
in modo che la traiettoria da seguire sia costantemente aggiornata in modo automatico.\\
Non è invece oggetto di questa parte di progetto la realizzazione del software per il drone, visto che questo verrà affrontato da terzi solo una volta dimostrata
la fattibilità del progetto.

\subsection{Analisi dei rischi}
Vista la totale non definizione della parte di machine learning e l'incarico di eseguire uno studio di fattibilità, viene preventivato il rischio di arrivare ad un non risultato.
In particolare potrebbe non essere identificato un dataset adatto, impedendo la realizzazione del sistema. La mitigazione di questo avviene grazie
all'incontro con un esperto in geologia pianificato a monte dello stage.\\
Un altro rischio è derivante dalla conoscenza ancora troppo superficiale circa i sistemi con intelligenza artificiale, sia personale che aziendale (visto che tale
topic non fa parte del business core), questo potrebbe causare la non identificazione di un modello appropriato per svolgere il compito richiesto.

\subsection{Aspettative aziendali}
Da questo progetto l'azienda si aspetta la concretizzazione di un'idea ancora molto indefinita. In particolare, è necessario comprendere se questo compito
è risolvibile con il machine learning o meno, e quale sia il modo migliore di raccogliere dati dal drone.\\
Non è richiesto un prodotto finito da mandare in un ambiente di produzione, tuttavia sarà necessario testare il modello prodotto per capirne l'affidabilità.\\
Facoltativamente è richiesto un front end per l'applicazione che permetta di selezionare la zona da sorvolare e che simuli il volo del drone, mostrando il cambiamento
di traiettoria in base a quanto classificato dal modello.

\subsection{Aspettative personali}
Da questa attività di tirocinio il candidato si aspetta di raggiungere un livello di autonomia nella progettazione e realizzazione di sistemi informatici più elevato,
riuscendo a fare scelte critiche riguardo alle tecnologie adottate in base alle necessità, visto che non sono stati imposti vincoli in questo senso.
Inoltre un obiettivo personale è acquisire capacità riguardo l'analisi ed elaborazione di un dataset ampio, capendo cosa comportino certe scelte su questo
all'intero sistema realizzato.

%**************************************************************
\section{Pianificazione e modalità di lavoro}
Le attività sono state svolte in totale autonomia in modalità remota, intervallate da incontri in sede con cadenza settimanale.\\
A questi erano presenti il CEO (nonché tutor esterno) e il CTO, il quale mi ha affiancato per tutta la durata del tirocinio verificando
costantemente il mio avanzamento.\\
Per quanto riguarda gli obiettivi obbligatori prefissati, si sono identificati i seguenti:
\begin{itemize}
		\item \textbf{OB1:} definizione della rappresentazione degli esempi per il modello di machine learning
		\item \textbf{OB2:} ricerca di un dataset sufficientemente ampio
		\item \textbf{OB3:} analisi del dataset e interpretazione dei dati
		\item \textbf{OB4:} realizzazione del modello di machine learning per la classificazione di zone
		\item \textbf{OB5:} analisi prestazionale del modello
		\item \textbf{OB6:} realizzazione di un web server per la predizione in real time
		\item \textbf{OB7:} realizzazione del programma per il calcolo della traiettoria, che si interfacci con il web server e preveda l'interfacciamento con il drone (per il progetto solo con un mock)
\end{itemize}
Ne segue questa pianificazione con cadenza settimanale, la quale ha subito diverse variazioni nel corso dello stage:
\begin{enumerate}
    \item Settimana dal 03/05 al 07/05
    \begin{itemize}
        \item analisi del problema;
        \item ricerca di un dataset adeguato;
    \end{itemize}
    
    \item Settimana dal 10/05 al 14/05
    \begin{itemize}
        \item analisi approfondita del dataset di firme spettrali USGS
        \item apprendimento del linguaggio python
    \end{itemize}
    
    \item Settimana dal 17/05 al 21/05
    \begin{itemize}
        \item rielaborazione del dataset con Matlab
        \item sviluppo del modello di machine learning su Google Colaboratory
        \item analisi prestazionale del modello
    \end{itemize}
    
    \item Settimana dal 24/05 al 28/05
    \begin{itemize}
        \item continuazione attività della settimana precedente
        \item esposizione del modello allenato su web server
    \end{itemize}
    
    \item Settimana dal 31/05 al 04/06
    \begin{itemize}
        \item progettazione e sviluppo business logic in linguaggio Java
        \begin{itemize}
            \item interfacciamento con web server Python per predizioni
            \item realizzazione di un web server in Java per l'interfacciamento con il drone
            \item realizzazione algoritmo per il calcolo della traiettoria
        \end{itemize}
    \end{itemize}
    
    \item Settimana dal 07/06 al 11/06
    \begin{itemize}
        \item continuazione attività della settimana precedente
        \item realizzazione unit test e integration test
    \end{itemize}
    
    \item Settimana dal 14/06 al 18/06
    \begin{itemize}
        \item continuazione attività della settimana precedente
        \item stesura documentazione del modello in Python per l'azienda
    \end{itemize}
    
    \item Settimana dal 21/06 al 25/06
    \begin{itemize}
        \item stesura documentazione business logic in Java
        \item presentazione del lavoro svolto e dei risultati ottenuti
    \end{itemize}
\end{enumerate}
Le attività qui elencate verranno meglio approfondite nel resto di questa tesi.