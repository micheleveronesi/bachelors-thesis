% !TEX encoding = UTF-8
% !TEX TS-program = pdflatex
% !TEX root = ../tesi.tex

%**************************************************************
\chapter{Introduzione}
\label{cap:introduzione}
%**************************************************************

\intro{Nel corso di questo capitolo si presenterà l'azienda in cui è stato svolto il progetto di stage,
portando inoltre l'attenzione sugli obiettivi, la pianificazione del lavoro e le modalità in cui si è svolto.}

%\noindent Esempio di utilizzo di un termine nel glossario \\
%\gls{api}. \\

%\noindent Esempio di citazione in linea \\
%\cite{site:agile-manifesto}. \\

%\noindent Esempio di citazione nel pie' di pagina \\
%citazione\footcite{womak:lean-thinking} \\

%**************************************************************
\section{Contesto aziendale}

L'azienda ospitante è denominata IM4DIGITAL s.r.l., società neonata gemella di Trizeta s.r.l.\\
Infatti condividono locali, risorse e obiettivi. In particolare le due aziende operano a stretto contatto con le realtà del territorio,
al fine di favorire l'automatizzazione di vari processi in settori diversi da quello informatico.\\
Trizeta è una software house di Padova specializzata in Consulenza, Integrated Solutions e IT Services, dedicata all’ideazione e progettazione “in-house” di soluzioni tecnologiche. Dal 2012 sviluppa internamente soluzioni tecnologiche per automatizzare i processi industriali e accompagnare le imprese in ogni fase della loro Transizione 4.0, dall’ideazione alla messa in opera dei progetti informatici.
L’offerta di Trizeta favorisce il successo dei clienti attraverso la realizzazione di soluzioni specifiche e alla consolidata esperienza dei suoi collaboratori.
Dal sistema di controllo e gestione aziendale ADeMES a quello del magazzino ADeWMS, passando per la consulenza da remoto attraverso la realtà aumentata ADeGO, Trizeta offre tutti gli strumenti e i servizi all’avanguardia per portare le aziende nel cuore dell’industria 4.0.

\begin{figure}[!h]
	\centering 
	\includegraphics[width=0.5\textwidth]{trizeta.png}
	\caption{Logo Trizeta}
\end{figure}

Il team è composto da poche persone, circa una decina. La metodologia di lavoro prediletta è quella agile, in quanto si cerca di intrattenere uno
stretto rapporto con i clienti. Inoltre questi vengono affiancati costantemente nel processo di digitalizzazione aziendale.

%**************************************************************
\section{Progetto di stage}

Lo scopo principale dello stage è uno studio di fattibilità, seguito dalla realizzazione di una Proof of Concept, per un sistema di navigazione
per droni senza pilota, il cui obiettivo è quello di sorvolare un campo di frutta identificando le zone in cui le piante soffrono di malattie e lo stato
di maturazione delle colture.\\
In particolare parte del sistema deve essere composta da un modello di machine learning per classificare il tipo di oggetti sorvolati dal drone,
in modo che la traiettoria da seguire sia costantemente aggiornata in modo automatico.\\
Non è invece oggetto di questa tesi la presentazione dell'interfacciamento con il drone, visto che questo verrà affrontato da terzi solo una volta dimostrata
la fattibilità del progetto.

\subsection{Analisi dei rischi}
Vista la totale non definizione della parte di machine learning e l'incarico di eseguire uno studio di fattibilità, viene preventivato il rischio di arrivare ad un risultato.
In particolare potrebbe non essere identificato un dataset adatto, impedendo la realizzazione del sistema. La mitigazione di questo avviene grazie
all'incontro con un esperto in geologia pianificato a monte dello stage.\\
Un altro rischio è derivante dalla conoscenza ancora troppo superficiale circa i sistemi con intelligenza artificiale, sia personale che aziendale (visto che tale
topic non fa parte del business core).

\subsection{Aspettative aziendali}
Da questo progetto l'azienda si aspetta la concretizzazione di un'idea ancora molto indefinita. In particolare, è necessario comprendere se questo compito
è risolvibile con il machine learning o meno, e quale sia il modo migliore di raccogliere dati dal drone.\\
Non è richiesto un prodotto finito da mandare in un ambiente di produzione, tuttavia sarà necessario testare il modello prodotto per capirne l'affidabilità.\\
Facoltativamente è richiesto un front end per l'applicazione che permetta di selezionare la zona da sorvolare e che simuli il volo del drone, mostrando il cambiamento
di traiettoria in base a quanto classificato dal modello.

\subsection{Aspettative personali}
Da questa attività di tirocinio mi aspetto di raggiungere un livello di autonomia nella progettazione e realizzazione di sistemi informatici più elevato,
riuscendo a fare scelte critiche riguardo alle tecnologie adottate in base alle necessità, visto che non sono stati imposti vincoli in questo senso.
Inoltre un obiettivo personale è acquisire capacità riguardo l'analisi ed elaborazione di un dataset ampio, capendo cosa comportino certe azioni sbagliate su questo
all'intero sistema realizzato.

%**************************************************************
\section{Pianificazione e modalità di lavoro}
Le attività sono state svolte in totale autonomia in modalità remoti, intervallate da incontri in sede con cadenza settimanale.\\
A questi erano presenti il CEO (nonché tutor esterno) e il CTO, il quale mi ha affiancato per tutta la durata del tirocinio verificando
costantemente il mio avanzamento.\\
Per quanto riguarda gli obiettivi obbligatori prefissati, si sono identificati i seguenti:
\begin{itemize}
		\item \textbf{OB1:} definizione della rappresentazione degli esempi per il modello di machine learning
		\item \textbf{OB2:} ricerca di un dataset sufficientemente ampio
		\item \textbf{OB3:} analisi del dataset e interpretazione dei dati
		\item \textbf{OB4:} realizzazione del modello di machine learning per la classificazione di zone
		\item \textbf{OB5:} analisi prestazionale del modello
		\item \textbf{OB6:} realizzazione di un web server per la predizione in real time
		\item \textbf{OB7:} realizzazione del programma per il calcolo della traiettoria, che si interfacci con il web server e preveda l'interfacciamento con il drone (per il progetto solo con un mock)
\end{itemize}